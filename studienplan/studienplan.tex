% ----------------------------------------------------------------------------
% 
% ----------------------------------------------------------------------------

\documentclass[11pt, parskip=half]{scrartcl}

%% Präambel
\usepackage[english, ngerman]{babel} % deutsche typogr. Regeln + Trenntabelle
\usepackage[T1]{fontenc}             % interner TeX-Font-Codierung
\usepackage{lmodern}                 % Font Latin Modern
\usepackage[utf8]{inputenc}          % Font-Codierung der Eingabedatei
\usepackage[babel]{csquotes}         % Anführungszeichen
\usepackage{graphicx}                % Graphiken
\usepackage{booktabs}                % Tabellen schöner
\usepackage{amsmath}      % Mathematik
\usepackage{amssymb}      % Mathematische Symbole
\usepackage[pdftex]{hyperref}       
\hypersetup{
	bookmarksopen=true,
	bookmarksopenlevel=3,
	colorlinks,
	citecolor=blue,
	linkcolor=blue,
}
\usepackage{scrhack} % unterdrückt Fehlermeldung von listings


\begin{document}

Niklas Deworetzki \hfill 16.05.2019

\begin{center}
  \begin{LARGE}
    Studienplan
  \end{LARGE}
\end{center}


Ich bin Informatikstudent im vierten Semester Bachelor Informatik.
Zum jetzigen Zeitpunkt habe ich alle Module in Regelstudienzeit bestanden.
Zwar habe ich im zweiten Semester das Modul ``Lineare Algebra'' schieben müssen, habe dieses Modul jedoch dann im dritten Semester zusätzlich belegt.
Im zweiten Semester habe ich zusätzlich ``Digitaltechnik'' belegt, da einer meiner Komillitonen mich auf das Modul aufmerksam gemacht hat und er es zur selben Zeit belegt hatte.
Während der Vorlesungsfreien Zeit zwischen dem Sommersemester 2018 und dem Wintersemester 2018/2019 habe ich an den Blockveranstaltungen ``Web Programming Weeks I'' und ``Generische Programmierung'' teilgenommen.

Seit meinem dritten Semester an der THM Gießen bin ich als Tutor angestellt.
Meine erste Anstellung war bei Herr Schölzel, wo ich an der JLU die Übungen im Modul ``Grundlagen der Informatik'' im Master Biologie betreut habe.
Im aktuellen Semester bin ich bei Herr Jäger angestellt als Tutor für das Modul ``Compilerbau'' im Bachelor Informatik.

Das Softwaretechnik-Projekt belege ich im aktuellen Semester bei Herr Kammer.
Der Schwerpunkt des Projektes liegt bei der Analyse der Daten, die von den Nutzern einer Website generiert werden.
Als Wahlpflichtveranstaltungen für das aktuelle Semester belege ich ``Funktionale Programmierung'' und ``Kategorientheorie''.

In Zukunft würde ich gerne meinen Schwerpunkt auf die theoretische Informatik und die funktionale Programmierung legen, wobei ich mich auch für die verschiedenen Programmierparadigmen generell und die Sprachentheorie interessiere.
Da ich durch Blockveranstaltungen schon viele Module des Wahlpflichtpools bestanden habe, weiß ich noch nicht, ob ich Wahlfplichtveranstaltungen im fünften Semester belegen werde.
Meine Entscheidung werde ich zur geeigneten davon abhängig machen, welche Module angeboten werden.
Die Tätigkeit als Tutor gefällt mir und ich würde sie gerne Fortsetzen.
Da meine Arbeit sowohl von Herr Schölzel als auch von Herr Jäger und Herr Meyer positiv bewertet wird, vermute ich jedoch, dass einer Weiterführung meiner Tutortätigkeiten nichts im Wege stehen wird.


 

\end{document}


%%% Local Variables:
%%% mode: latex
%%% TeX-master: t
%%% End:
